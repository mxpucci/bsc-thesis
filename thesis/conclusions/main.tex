\chapter{Conclusions}

In this thesis, we have proposed and rigorously evaluated novel compression techniques specifically tailored for static and slightly dynamic labeled property graphs (LPGs). Our contributions focus on significantly reducing storage requirements while maintaining, and in many cases improving, query performance over existing solutions.

We introduced an innovative method for compressing the graph structure by proposing a new compressed indexing that applies to every integer sequence. We also leveraged permutation-based and FM-index-based mappings to represent node identifiers efficiently. These methods demonstrated substantial improvements in compression ratios and competitive performance across both sparse and dense graph structures. By carefully selecting node relabeling strategies, particularly traversal-based techniques such as the Cuthill-McKee ordering, we effectively minimized graph bandwidth, thereby directly enhancing our compression efficiency.

Furthermore, we addressed property storage by designing specialized encoding schemes for fixed-size (integers, dates, doubles, enums) and variable-size (strings) node and edge attributes. The type-aware compression methodology developed significantly reduces the space overhead associated with commercial graph databases, preserving rapid property retrieval capabilities. Our experimental results demonstrated a reduction in storage requirements of up to 98\% compared to conventional graph databases, and an improvement of approximately 27\% compared to other unlabeled graph compressors.

Our extensive experimental evaluation, conducted on diverse datasets including biological, academic, e-commerce, and patent citation networks, underscored the versatility and robustness of our approach. Our benchmarks provided clear evidence that our compressed representation of a labeled graph not only achieves superior space efficiency but also delivers highly competitive query performance for common graph operations such as neighbor queries and property retrievals.

These achievements affirm the viability of our computation-friendly compression design, particularly in contexts where expressive, standards-compliant graph query engines (GQL-enabled) are necessary. By allowing query operations to execute efficiently directly on the compressed representation, our work sets a solid foundation for future research and development.

Looking forward, potential avenues of exploration include extending our compression techniques to fully dynamic graphs and exploring parallelization strategies to optimize build and load times. Furthermore, assessing our methodologies in more specialized contexts, such as real-time analytics or knowledge graph integration in retrieval-augmented generation systems, may yield insightful and practical advancements.