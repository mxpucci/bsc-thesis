\section{Patents: american patent citation graph}
This section describes part of a project conducted by the National Bureau of Economic Research (NBER) \cite{patentsource} on American patent data from 1976 to 2006. The project encompasses details on over 3 million patents, including more than 16 million citations among them\footnote{The raw data is available at \url{https://sites.google.com/site/patentdataproject/Home/downloads/patn-data-description?authuser=0}}.

In this graph, the nodes represent either patents or assignees. Patent nodes are interconnected by edges of type \texttt{CITES}. Additionally, corresponding relationships link patents to their assignees—entities, companies, or individuals.\footnote{The processed dataset can be downloaded from \url{https://drive.google.com/file/d/1mCov-nqQC0BE-gGBf7NXq4YRRG9PEnDh/view?usp=sharing}.}

The entire dataset occupies 233 MB of storage when compressed in a ZIP archive, containing approximately 3.5 million nodes and 26.4 million edges. In its uncompressed form, the nodes file and the edges file require approximately 338 MB and 905 MB, respectively.

All information about categories and subcategories is based on the classification provided by the Commerce Control List (CCL).

\subsection*{Node Properties}
(Almost all properties are set only for nodes of type \texttt{patent} unless explicitly stated otherwise.)
\begin{itemize}
    \item \emph{id}: Unique identifier for each node.
    \item \emph{type}: Indicates whether the node is a \texttt{Patent} or an \texttt{Assignee}.
    \item \emph{name}: (For nodes of type \texttt{Assignee}) The name of the assignee.
    \item \emph{Cat}: Technological category according to the CCL, ranging from 1 to 6.
    \item \emph{Cclass}: United States Patent Classification (USPC) code that categorizes the patent.
    \item \emph{Status}: Authorization status of the patent (where \texttt{m} indicates missing and \texttt{w} indicates withdrawn).
    \item \emph{Subcat}: CCL-related subcategory (values between 11 and 69).
    \item \emph{Subclass}: Subclass for the given USPC class (as defined by the CCL).
    \item \emph{subclass1}: Similar to \texttt{Subclass}, but referring to the alpha-subclasses.
    \item \emph{icl}: International classification.
    \item \emph{icl\_class}: A 4-character code specifying the class according to the International Patent Classification (IPC).
    \item \emph{icl\_maingroup}: The main group containing the IPC class.
    \item \emph{Gday}: Day on which the patent was granted.
    \item \emph{Gmonth}: Month in which the patent was granted.
    \item \emph{Gyear}: Year in which the patent was granted.
    \item \emph{Appyear}: Year in which the patent was submitted.
    \item \emph{Nclaims}: Number of claims in the patent.
    \item \emph{Iclnum}: Sequence number associated with the IPC class.
    \item \emph{Nclass}: A three-digit number specifying the patent class according to the CCL.
    \item \emph{term\_extension}: Number of days the patent term was extended.
    \item \emph{allcites}: Total number of citations received by the patent up to 2006. Note that this number is imprecise for patents with submission years between 2006 and 2009; a secondary field, \texttt{hjtwt}, is provided to correct this inaccuracy.
    \item \emph{hjtwt}: Correction value for \texttt{allcites}.
    \item \emph{uspto\_assignee}: (For nodes of type \texttt{Assignee}) The original identification number of the assignee as provided by the United States Patent and Trademark Office (USPTO).
\end{itemize}

\subsection*{Edge Properties}
\begin{itemize}
    \item \emph{source}: The source node.
    \item \emph{target}: The destination node.
    \item \emph{type}: Indicates the type of relationship, e.g., \texttt{ASSIGNED\_TO} or \texttt{CITES}.
\end{itemize}